\documentclass[12pt,a4paper]{article}
\makeatletter

\def\id#1{\def\@id{#1}}
\def\department#1{\def\@department{#1}}

\def\@maketitle{
\begin{center}
\vspace{10mm}
{\LARGE\bf \@title \par}
\vspace{10mm}
{\Large \@date\par}
\vspace{20mm}
{\Large \@department \par}
{\Large 学籍番号 \@id \par}
\vspace{10mm}
{\large \@author}
\end{center}
\par\vskip 1.5em
}

\makeatother

\title{代数学1 レポート}
\author{瀧ヶ平 充}
\department{東京理科大学 理学部第一部 応用数学科 4年}
\id{1414059}

\usepackage[truedimen,margin=25truemm]{geometry}
\usepackage{amsmath}

\begin{document}
  \begin{titlepage}
    \maketitle  
  \end{titlepage}

\section*{問題1}
  $f(X) = X^4 + 4X^3 - 4X^2 + X -3$に対して、
  \begin{enumerate}
    \item $f(X)$のスツルム列を計算せよ。
    \item $f(X)$の$0 < X \leq 2$における実根の個数を求めよ
    \item $f(X)$の$2 < X \leq 4$における実根の個数を求めよ
  \end{enumerate}

  \subsection*{解}

    $f_0(X) = f(X), f_1(X) = f_0^\prime(X) = 4X^3 + 12X^2 - 8X + 1$で、\\
    $f_0(X) = \frac{1}{4}(X + 1)f_1(X) - 5X^2 + \frac{11}{4} X - \frac{13}{4}$より$f_2(X) = - 5X^2 + \frac{11}{4} X - \frac{13}{4}$\\
    $f_1(X) = - \frac{4}{5}(X + \frac{71}{5})f_2(X) - \frac{279}{100} X - \frac{823}{100}$より$f_3(X) = - \frac{279}{100} X - \frac{823}{100}$\\
    $f_2(X) = \frac{1}{279}(500X + \frac{488225}{279})f_3(X) -\frac{4271075}{77841}$より$f_4(X) = -\frac{4271075}{77841}$\\
    $f_0(0) < 0, f_1(0) > 0, f_2(0) < 0, f_3(0) < 0, f_4(0) < 0$より$V(0)=2$\\
    $f_0(2) > 0, f_1(2) > 0, f_2(2) < 0, f_3(2) < 0, f_4(2) < 0$より$V(2)=1$\\
    $f_0(4) > 0, f_1(4) > 0, f_2(4) < 0, f_3(4) < 0, f_4(4) < 0$より$V(4)=1$\\
    よって$f(X)$の$0 < X \leq 2$における実根の個数は$V(0) - V(2) = 1$個\\
    よって$f(X)$の$2 < X \leq 4$における実根の個数は$V(2) - V(4) = 0$個

\newpage

\section*{問題2}
  \subsection*{(1)}

    2つの有理数$\frac{p_1}{q_1},\frac{p_1}{q_1}$にたいして、有理数$f_1 (\frac{p_1}{q_1},\frac{p_1}{q_1})$を以下のように定義する。

    $f_1(\frac{p_1}{q_1},\frac{p_1}{q_1}) = \frac{p_1}{q_2} + \frac{p_2}{q_1}$

  \subsection*{(2)}
    
    空でない整数の有限集合$A = \lbrace a_1, \dots ,a_m \rbrace B = \lbrace b_1, \dots ,b_n \rbrace$にたいして、\\
    整数の有限集合$f_2(A,B)$を以下のように定義する。
    
    $f_2(A,B) = \lbrace a_1 + b_1 , \dots , a_1 + b_n , a_2 + b_1 , \dots , a_2 + b_n , \dots , a_m + b_1 , \dots , a_m + b_m \rbrace $
  
  \subsection*{(3)}
    
    空でない整数の有限集合$A = \lbrace a_1, a_2 \rbrace B = \lbrace b_1, b_2 \rbrace$にたいして、\\
    整数の有限集合$f_3(A,B)$を以下のように定義する。

    $f_3(A,B) = \lbrace a_1 + b_1, a_2 + b_2 \rbrace$

  \subsection*{解}
    $f_1$に対し、$f_1(\frac{1}{2},\frac{1}{3}) = \frac{1}{3} + \frac{1}{2} \neq f_1(\frac{2}{4}, \frac{1}{3}) = \frac{2}{3} + \frac{1}{4}$\\
    これは、$\frac{1}{2} = \frac{2}{4}$に反するため、well-definedではない。

    \noindent
    $f_2$に対し、$A$を異なる順序に入れ替えたの$A^\prime$は$A \equiv A^\prime$で、任意の$B$に対して、$f_2(A, B) \equiv f_2(A^\prime, B)$となり、$B$を異なる順序に入れ替えた$B^\prime$を考えるときも同様のことが言える。\\
    よって、$f_2$はwell-defined。

    \noindent
    $f_3$に対して、 $A = \lbrace a_1, a_2 \rbrace \equiv \lbrace a_2, a_1 \rbrace$で、$A^\prime = \lbrace a_2, a_1 \rbrace$とすると\\
    $f_3(A, B) = \lbrace a_1 + b_1 , a_2 + b_2 \rbrace \neq \lbrace a_2 + b_1 , a_1 + b_2 \rbrace = f_3(A^\prime, B)$となり、$A \equiv A^\prime$に矛盾する。\\
    よって$f_3$はwell-definedではない。


\end{document}
